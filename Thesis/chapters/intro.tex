\chapter{引言}
在数字时代,操作系统的重要性不言而喻,它作为计算机软硬件之间的桥梁,存在于日常生活的每一个角落,
人们的衣食住行都无法与各种各样的系统完全脱离。

操作系统在各种应用场合向各种应用程序提供运行环境:
\begin{description}
    \item [办公] 无纸化办公已经成为现代企业必备,各种文档都保存到公司的电脑和服务器中,
    这些终端保存文档的过程离不开操作系统;企业办公必备的Email,打卡等也都必须依赖于操作系统;
    \item [移动] 看新闻离不开新闻APP,点外卖离不开外卖APP,
    连看时间和设置闹铃的时钟也都需要运行在操作系统环境上;
    \item [出行] 出行方式包括车辆,船只,飞机等,而现代调度平台已经全部电子化,即调度工作依赖于操作系统。
\end{description}

可见,操作系统在现代社会的必要性,而作为一个被依赖如此严重的软件,其设计的复杂程序是常人难以想象的,
通常只有庞大的互联网科技公司聚集成百上千的高级工程师花费无法计数的日月才能完成一个操作系统,
而设计并完成一个操作系统对于一个学生而言是一个几乎不可能完成的挑战,
但是想要提高理论与技术能力就必须克服难关,勇敢的跨出这一步,
所以设计并完成一个基本满足日常功能需求的操作系统作为此次的目标,
并以此为跳板对操作系统进行更深一步的探究。