\chapter{技术路线}

此次的技术路线由3步逐步进行:
\begin{enumerate}
    \item 操作系统探究
    \item 编写操作系统内核
    \item 对外兼容及安全防护
    \end{enumerate}

\section{操作系统探究}
操作系统的发展是跟随计算机性能的提升一步步趋于完善的,
从历史上第一台计算机的诞生直到今天的计算机,
计算机的性能从处理字节级别到胜任各种高级任务,
操作系统的功能也从手动调度计算机硬件到了操作系统自动调度各种高级程序使用相关硬件,
并对用户需求的各项功能提供接口。

随着计算机走入寻常百姓家,计算机需要面对的生活需求越来越多,
计算机操作系统随之发展以适应各种工作环境的各种工作需求,
这样的发展态势使得操作系统的功能越来越多,各项功能也逐渐趋于完善,
但是操作系统出现问题的概率也随着代码数量的增加而增加,
联系到人们的需求,寻求符合操作系统发展且适应用户使用的功能,
并将其他功能作为可选项排除在操作系统内核之外已经越来越成为趋势。

\section{编写操作系统内核}
从探索的结果开始设计并实现功能操作系统功能。
\begin{description}
    \item[空白操作系统启动:]
    计算机用户平时启动计算机操作是点按电源键,然后等待计算机开机并出现习惯的操作系统界面出现,并开始借助计算机完成各项工作。
    对普通用户而言按了开机键后操作系统启动是理所应当的,但是其中的过程普通用户看不见,是完全封闭的黑匣子。
    打开这个黑匣子,利用操作系统相关知识并查阅相关资料探究操作系统从电气设备到软件代码的衔接则是探究操作系统的第一步。

    \item [内存管理:]
    内存是现代计算机的最基础的部件之一,程序的运行必须加载到内存中,所以在完成操作系统启动的下一步就是处理好内存的管理问题。
    内存的管理涉及到内存空间的初始化,分配和释放问题,首当其冲的是对内存空间的初始化,即按照一定的大小将内存分割成若干份。
    然后对申请内存的程序按照其需求大小分配内存空间,在程序运行结束后释放其内存空间。
    
    好的内存管理方式对计算机运行效率至关重要:
    \begin{enumerate}
        \item 如果初始化内存指定大小过大将导致多程序运行时每个程序占用内存过剩,无空闲内存给新程序;
    初始化大小过小则导致内存管理程序任务过重,需要花费过多的资源用于管理内存;
        \item 如果分配过程中将内存分配过于分散也将导致管理内存程序花费过多资源;
        \item 在释放内存空间的过程中,算法设计不合理将导致磁盘空间碎片化严重,
        而如果算法错误导致程序出错指针等指向错误甚至可能导致系统崩溃。
    \end{enumerate}

    \item [输入输出:]
    计算机的出现是为了处理一些用户不想处理甚至无法处理的任务。
    所以计算机与用户的交互在计算机运行过程中也很重要,
        计算机的运行任务是用户给予的,计算机的运行结果也是为了给用户的。
    用户与计算机进行数据交流的最基础方式是键盘输入,而后也出现了鼠标的图形化输入。
    键盘输入的精准性一直被世人所称赞,而鼠标的空间方式也为图形化界面用户提供了方便。    

    \item [多进程及分时系统:]
    初代计算机用户只需要运行一个需要花费大量CPU运行时间的数字运算,
    而现代计算机用户对计算机需求却离不开多任务。

    多任务常出现在生活中,一个人工作时会抽出时间帮他人一个小忙又继续自己的工作,
    但是计算机没有人类复杂的大脑,计算机同时只能完成一个任务。

    所以在现代计算机操作系统发展中里程碑式出现了多道程序设计和分时操作系统,
    在相当程度上完成了本不可能出现了“同时运行多个程序”,
    使多个程序在人类不能明显感觉到的时间间隔内来回切换,
    并在切换过程中保存程序运行现场保证下一次程序可以继续运行。

    现代计算机多任务的基础技术,对计算机操作系统的发展至关重要。
\end{description}

\section{实现对外接口及安全防护}

计算机操作系统的功能按常理来说是可以无限扩展的,但是扩展将导致操作系统无限臃肿,
而后果是操作系统的维护无限困难,更难以接受的是臃肿的系统将出现无数漏洞暴露给恶意程序。

所以在不削减操作系统功能的基础上,将操作系统分为内核和外部应用程序,
操作系统内核为外部应用程序提供接口,以指定的方式让外部应用程序调用系统函数间接完成对硬件的操作,
在系统易于维护且保证安全的基础上为用户提供需要的功能。

系统的安全除了对系统接口的授权外,也涉及到监控无法正常运行的程序并使其终止,对损坏硬件的检测等方面。