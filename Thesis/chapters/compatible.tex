\chapter{实现对外兼容及安全防护}

从接口设计及安全防护的角度完善操作系统

\section{系统调用}

操作系统是用于管理硬件的软件,但是对绝大部分人都不算友好,
可友好不是操作系统必须的,操作系统只需要快速和高效,
为此操作系统向外界开放接口,
由其他的编程人员来使用接口完成各种功能的实现,
比如:图形界面,各种字处理软件等。

系统调用的实现是通过程序向操作系统申请权限访问各种指定的系统函数,
操作系统操作硬件完成工作。

向外界开发的系统调用接口:
\begin{enumerate}
    \item 显示单个字符
    \item 显示字符串
    \item 键盘输入
    \item 定时器
    \item 文件操作
    \item 命令行
\end{enumerate}

\section{系统安全}

并不是所有程序都是友好的,开发系统接口的风险就是操作系统不再像之前那么安全,
一旦有程序跨越操作系统直接操作硬件很可能使得操作系统崩溃。

所以操作系统在开放的同时也需要一定的安全举措:
\begin{enumerate}
    \item 内存写保护
    \item 系统函数权限
\end{enumerate}