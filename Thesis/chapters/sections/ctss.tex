\section{分时操作系统}

在上一节中说到分时是使得在用户看来计算机的多道程序同时运行,多道程序已经实现了,
分时简单说是使得CPU在用户不能明显感觉到的时间间隔内切换运行多个程序,
在切换后每个程序都能对作业进行一定的处理,在进行多个周期后,各个程序先后完成作业。

不能明显感觉到的时间间隔内切换中有两个概念,时间间隔和切换:

时间间隔太长则用户会有明显的卡顿感,不利于分时概念的实现,
而间隔时间太短则时间不足以让程序响应并完成一定量的工作,同样不利于分时概念的实现。

切换涉及到保存当前程序的运行状态以便于程序获得时间片后可以接续上次的任务继续执行。

