\section{多道程序系统}

现代的计算机已经不仅仅作为数字计算的工具,而进入大众生活的计算机被赋予了更多的生活需求,
用户可能在看电影的同时查看电子邮件,也有可能在写论文的时候进入浏览器查询相关资料,
但是更重要的是计算机往往在用户不经意间打开防病毒软件等保证用户计算机的安全。

由此可见多进程的工作方式在计算机工作中不可缺少。

但是在实际的处理过程中,计算机并不能同时处理多个程序,所以必须采用分时的设计,
关于分时操作系统的设计在下一节。
在此有两个概念,同时处理和多个程序,同时处理属于分时,多道程序属于多道程序设计。

首先要处理的问题是如何运行多个程序,早期的多道程序设计的出发点是充分的利用CPU,
作为输入设备的打孔纸带与CPU速度相比差距过大,昂贵的CPU常常在等待I/O信号而闲置。

具体情形如:A作业等待磁盘或者其他I/O时,CPU暂停为A作业服务,而转向为已I/O操作已完成的B作业服务。

由于A作业和B作业并行存储在计算机内存中,操作员无法区分现在是正在运行的是A作业还是B作业,
但两个作业输出在同一个磁盘,故认为是多道程序在计算机中运行。

