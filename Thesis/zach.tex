\documentclass{swfcthesis}

\addbibresource{thesis.bib}    % 参照教程自己去写一个.bib文件

\begin{document}

\Title{操作系统探索}
\Author{尹志成}
\Advisor{王晓林}
\AdvisorTitle{讲师}
\AdvisorInfo{王晓林,男,49 岁,硕士,讲师,毕业于英国格林尼治大学,分布式计算系统专业。现任西南林业大学计信学院教师。执教 Linux、操作系统、网络技术等方面的课程,有丰富的 Linux 教学和系统管理经验。}
\Month{六}
\Year{二〇一八}
\Subject{计算机科学与技术专业}    %专业名称(比如 计算机科学与技术专业)
\Abstract{操作系统最初的诞生是为了搭配进行简单繁重的数字运算机,但随着时代的演进,计算机不仅作为处理各种运算的机器,其附加价值也越来越被人们看重,跟随着计算机的发展,操作系统的使命也在一代代的改变,(约两百字)}
\Keywords{操作系统}
\Acknowledgments{感谢,}
\enTitle{Operating system exploration}
\enAuthor{Zach Yin}
\enAbstract{英文摘要}
\enKeywords{Operate System}

%%% 下面六行不要动!
\makepreliminarypages% 封面
\frontmatter          
\tableofcontents     % 目录
\listoffigures       % 插图目录
\listoftables        % 表格目录
\mainmatter

\chapter{绪论}
在数字时代,操作系统的重要性不言而喻,它作为计算机软硬件之间的桥梁,存在于日常生活的每一个角落,而研究一个只有庞大的公司聚集成百上千的高级工程师才能完成的操作系统对于学生而言是一个几乎不可能完成的挑战,但是克服难关是锻炼技术的必经之路\cite{30_os},所以研究并完成一个基本满足日常功能需求的操作系统作为此次的目标,并以此为跳板对操作系统进行更深一步的探究。
\chapter{思路}
	此次的思路由四部分组成:
	1、操作系统探究
	2、空白操作系统的启动
	3、编写操作系统内核
	4、实现对外兼容及安全防护
	
	\section{操作系统探究}
	从历史上计算机操作系统的发展联系到人们的日常生活,寻求符合操作系统发展且适应用户使用的特征要素。
	
	\section{空白操作系统的启动}
	利用汇编语言及操作系统相关知识探究操作系统如何从电气设备到软件代码的衔接
	
	\section{丰富操作系统内容}
	从内存管理,输入输出,多进程,分时四个模块丰富操作系统的内容
	
	\section{实现对外接口及安全防护}
	从接口设计及安全防护的角度完善操作系统
	
\chapter{操作系统探索}

\chapter{空白操作系统的启动}

	\section{操作系统启动流程}
	按下电源键之后启动计算机,启动过程分为四个阶段\cite{hbt}:
		\begin{center}BIOS -> MBR -> VBR -> 操作系统\end{center}
		
		1、在BIOS完成POST(硬件自检,Power-On Self Test)并选择启动顺序(Boot Sequence)把控制权转交给排在第一位的储存设备
		
		2、计算机读取该设备的MBR(第一个扇区,最前面的512个字节),在此装入ZOS的启动程序IPL(Initial Program Loader)程序ipl09.nas
		
		\hspace*{1cm}ipl09.nas指明了操作系统的位置,主分区第一个扇区的物理位置(柱面、磁头、扇区号等等)

		3、计算机根据VBR(Volume boot record)指引得到操作系统在这个分区里的位置继而加载操作系统
		
		4、控制权转交给操作系统后,操作系统的Kernel被载入内存
		
	\section{制作MBR(ipl09.nas)}
		MBR负责指出操作系统的位置,主分区第一个扇区的物理位置(柱面、磁头、扇区号等等)
		\inputminted[tabsize=2, firstline=6, lastline=6,
		linenos=true]{nasm}{../ZOS/src/kernel/ipl09.nas}
		
		\inputminted[tabsize=2, firstline=12, lastline=29,
		linenos=true]{nasm}{../ZOS/src/kernel/ipl09.nas}
		
		\inputminted[tabsize=2, firstline=43, lastline=45,
		linenos=true]{nasm}{../ZOS/src/kernel/ipl09.nas}
		
		\inputminted[tabsize=2, firstline=76, lastline=88,
		linenos=true]{nasm}{../ZOS/src/kernel/ipl09.nas}
	
	\begin{listing}[H]
		\inputminted[tabsize=2, firstline=125, lastline=147,
		linenos=true]{nasm}{../ZOS/src/kernel/ipl09.nas}
	\end{listing}

	\section{制作空白操作系统}
	首个操作系统为测试操作,目的是测试MBR可以成功启动操作系统
	\begin{minted}{c}
			fin:
			    HLT
			    JMP fin
	\end{minted}
		
\chapter{编写操作系统内核}
	从内存管理,输入输出,多进程,分时四个模块丰富操作系统的内容
	\section{内存管理}
		内存 (RAM) 是计算机中不可或缺的重要硬件,所有程序的运行都是在内存中进行的,而CPU访问硬盘数据也必须先经过内存交换才得以实现,内存在加速CPU访问硬盘居功至伟。
		由内存的重要性可知内存管理在操作系统中也非常重要。	
		
		内存管理设计的主要目的是快速并且高效的分配内存空间,并在适当的时间释放并回收内存空间。
		根据内存管理的设计目的,内存管理的数据结构如下:
		\inputminted[tabsize=2, firstline=137, lastline=143,
		linenos=true]{c}{../ZOS/src/kernel/bootpack.h}
		
		\begin{description}
		\item[frees:]可用信息数目
		\item[maxfrees:]用于观察可用状况:frees的最大值
		\item[lostsize:]释放失败的内存的大小总和
		\item[losts:]释放失败次数
		\end{description}
		
		经过内存初始化和释放所有内存空间后,内存管理正常运行。
		
		\subsection{内存分配}

		\begin{listing}[H]
		\inputminted[tabsize=2, firstline=68, lastline=80,
		linenos=true]{c}{../ZOS/src/kernel/memory.c}
		\end{listing}

		memory
		\subsection{内存释放}
		为保证磁盘空闲空间尽可能不呈现碎片化,内存释放主要分为三种情况:
		\begin{description}
			\item[前端可用:]释放内存的相连前端是空闲内存或释放内存相连两端都是空闲内存
			\item[后端可用:]释放内存的相连后端是空闲空间
			\item[前端后端均不可用:]挪动空闲空间以
		\end{description}

		前端可用:
		\begin{listing}[H]
		\inputminted[tabsize=2, firstline=97, lastline=117,
		linenos=true]{c}{../ZOS/src/kernel/memory.c}
		\end{listing}

		后端可用:
		\begin{listing}[H]
		\inputminted[tabsize=2, firstline=118, lastline=127,
		linenos=true]{c}{../ZOS/src/kernel/memory.c}
		\end{listing}

		前端后端均不可用
		\begin{listing}[H]
		\inputminted[tabsize=2, firstline=128, lastline=141,
		linenos=true]{c}{../ZOS/src/kernel/memory.c}
		\end{listing}

	\section{输入输出}
	\section{多进程}
	\section{分时}
	
\chapter{实现对外兼容及安全防护}
从接口设计及安全防护的角度完善操作系统


\chapter{另一章}

\section{图片与表格}

如果需要插入图片与表格的话,可以参考下面的简单例子。

\subsection{图片示例}

下面是插入图片的示例:

\begin{figure}[!ht]
  \centering
  %\includegraphics[width=.5\textwidth]{hello}
  \caption{图片示例}
  \label{fig:hello}
\end{figure}

\subsection{表格示例}

下面是一个表格的例子:

\begin{table}[!ht]
  \centering
  \begin{tabular}{|r|c|l|}    \hline
    Hello&world&Hello, world!\\hline
    Hello&world&Hello, world!\\hline
  \end{tabular}
  \caption{表格示例}
  %\label{tab:hello}
\end{table}

\chapter{又一章标题}

接着写吧接着写吧接着写吧接着写吧

%%% 正文部分到此结束。下面是『参考文献』、『指导教师简介』、『鸣谢』、『附录』

%% 不要动下面四行!
\Appendix{}
\printbibliography[heading={bibintoc},title={参考文献}] % 输出参考文献
\advisorinfopage{}                 % 输出指导教师简介
\acknowledgmentspage{}             % 输出鸣谢

%%% 下面是附录部分,可以没有。

\chapter{我也不知道为什么要写附录} %附录一

可以参考模版目录中的 appendix.tex 文件来写。

\chapter{主要程序代码} %附录二

% 插入程序代码
%\inputminted[fontsize=\small]{c}{hello.c}

% 也可以这样
\begin{listing}[H]
  %\inputminted{c}{hello.c}
  %\caption{Hello, world!}
  %\label{lst:hello}
\end{listing}  

\end{document} % 结束。不要动下面几行!

%%% Local Variables:
%%% mode: latex
%%% TeX-master: t
%%% End:
