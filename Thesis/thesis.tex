\documentclass{swfcthesis}

\addbibresource{bb.bib}

\begin{document}

\Title{操作系统探索}
\Author{尹志成}
\Advisor{王晓林}
\AdvisorTitle{讲师}
\AdvisorInfo{王晓林,男,49 岁,硕士,讲师,毕业于英国格林尼治大学,分布式计算系统专业。
  现任西南林业大学大数据与只能工程学院教师。执教 Linux、操作系统、网络技术等方面的课程,有丰富的 Linux 教学和系统管理经验。}
\Month{六}
\Year{二〇一八}
\Subject{计算机科学与技术专业}
\Abstract{
  操作系统最初的诞生是为了搭配进行简单繁重的数字运算机,但随着时代的演进,
  计算机不仅作为处理各种运算的机器,其附加价值也越来越被人们看重,随着计算机的发展,
  多道程序系统,分时操作系统,一个个新技术的诞生标志着操作系统越发的完善,
  但是为了探究操作系统,那就不应该满足现有的操作系统,
  应该拨开操作系统华丽的外衣,从底层开始,经过一步步的摸索,不断探索并实现各项功能,
  从最简洁的代码完成一个满足现代各项需求的操作系统。
  }
\Keywords{操作系统\ 设计\ 原理\ 简化}
\Acknowledgments{
  四年时光匆匆而过,解了不多不少的惑。一初听说最怕的人,成了感谢最多的人。

  当过班上班主任的王晓林老师,在我心里仍然是班主任,我想这不止是因为他当过,而是我认为他当过。
  他的认真负责和较真较劲让上过课的每个学生印象深刻,
  他教书教得好,是大家都认同的,而这里的大家还包括一些讨厌他古板作派的同学。
  他的一门课只有一个presentation,学期结束了,presentation也就结束了,
  presentation的内容不跟教学大纲走,在大学课本里也都找不到,
  但是总能在权威的技术书籍里看到影子,也总能在相关知识的学习中联想到并且用到。
  他上课不喜欢教,就喜欢问,然后像个学生一样跟学生讨论来讨论去,
  讨论着讨论着我们就学会了,而且印象深刻。
  他崇尚GNU的自由精神,也不断告诉我们觉得对的就应该不顾一切勇敢的努力的去做。
  他也喜欢干点专权的事,拿着老师的架子逼着全班学生装Linux,也许之前也想不通为什么要这样做,
  但是如果没有这样一个人逼着改变,我想Linux对我只是一门课,而现在Linux是我提高效率的工具。
  
  选择毕业设计导师的时候我没怎么犹豫,我想,对,就是他。
  }
\enTitle{Operating system exploration}
\enAuthor{Zach Yin}
\enAbstract{
  The initial version of the operating system was to collaborate with simple and heavy-duty digital computing machines. 
  However, with the evolution of the times, computers have not only been used as machines for processing various operations, 
  but also added more value with people. With the development of computers, The multi-programming system, 
  the time-sharing operating system, and the emergence of new technologies mark the increasingly perfect operating system. 
  However, to explore the operating system, it should not satisfy the existing operating system. 
  It should open up the gorgeous operating system. Starting from the ground floor, 
  after a step-by-step exploration, we constantly explored and implemented various functions, 
  completing an operating system that satisfies modern requirements from the most concise code.
  }
\enKeywords{OperatingSystem\ Design\ Principle\ simplify}

%%% 下面六行不要动!
\makepreliminarypages% 封面
\frontmatter          
\tableofcontents     % 目录
\listoffigures       % 插图目录
\listoftables        % 表格目录
\mainmatter

% 引言 v
\chapter{绪论}
在数字时代,操作系统的重要性不言而喻,它作为计算机软硬件之间的桥梁,存在于日常生活的每一个角落,
而研究一个只有庞大的公司聚集成百上千的高级工程师才能完成的操作系统对于学生而言是一个几乎不可能完成的挑战,
但是克服难关是锻炼技术的必经之路\cite{30_osKawaiHidemi200630},所以研究并完成一个基本满足日常功能需求的操作系统作为此次的目标,
并以此为跳板对操作系统进行更深一步的探究。

% 思路 v
\chapter{思路}

此次的思路由四部分组成:\\
\hspace*{1.5cm}1、操作系统探究 \\
\hspace*{1.5cm}2、空白操作系统的启动 \\
\hspace*{1.5cm}3、编写操作系统内核 \\
\hspace*{1.5cm}4、实现对外兼容及安全防护

\section{操作系统探究}
从历史上计算机操作系统的发展联系到人们的日常生活,寻求符合操作系统发展且适应用户使用的特征要素。

\section{空白操作系统的启动}
利用汇编语言及操作系统相关知识探究操作系统如何从电气设备到软件代码的衔接

\section{完善操作系统内核}
从内存管理,输入输出,多进程,分时四个模块丰富操作系统的内核

\section{实现对外接口及安全防护}
从接口设计及安全防护的角度完善操作系统

% 原理 第四代第五代 结论
\chapter{操作系统探索}

\section{操作系统的诞生}

\subsection{第一代:真空管}

操作系统最初出现的场景是一个工程师小组设计、建造一台机器,之后使用机器语言编写程序并通过将上千根电缆接到插线板上连接成电路,
控制机器的基本功能,进而操作机器运算诸如制作正弦、余弦、对数表或计算炮弹弹道的简单数学运算。

这里的人工拔插电缆就充当着操作系统的角色——根据程序直接操作硬件使其运算得出结果。

% -------------

\subsection{第二代:晶体管}

在晶体管发明后,计算机可靠程度大大增加,计算机开始被一些公司、政府部门或大学使用。

改进后出现了操作系统的载体,卡片和较后期磁带打孔纸带。

由于打孔纸带是分次读入,一次只能读入一个的作业,出现了批处理系统如图~\ref{fig:btss}: 

\begin{figure}[h]
  \centering
  \includegraphics[width=.8\textwidth]{fig/btss.png}
  \caption{批处理系统}
  \label{fig:btss}
\end{figure}

\begin{description}
\item[a.]程序员将打孔纸带拿到1401机\footnote{IBM 1401:数据处理计算机\cite{1401dps}}
\item[b.]1401机将批处理作业读到磁带上
\item[c.]操作员将输入磁带送至7094机\footnote{IBM 7094:专为大型科学计算而设计,具有出色的性价比和扩展的计算能力\cite{7094dps}}
\item[d.]7094机进行计算
\item[e.]操作员将输出磁带送到1401机
\item[f.]1401机打印输出
\end{description}

在这里,操作系统的工作已经由完全的人工转换到一部分人工操作交由机器完成,工作效率较之前大大提高,
并且,由于加入了磁带,计算机完成的工作也将及时得到保存。
操作系统的工作已经开始有一定的流程化了。

% -------------

\subsection{第三代:集成电路}

采用集成电路的第三代计算机较分立晶体管的第二代计算机在性能/价格比上有了很大的提高,
第三代操作系统加入了多道程序设计和分时系统,可以适应多道程序同时运行的任务。
\begin{enumerate}
  \item 多道程序设计主要目的是解决CPU因等待磁带或其他I/O操作而暂停工作,多道程序设计可以使CPU在程序a的I/O操作时运行程序b\cite{tanenbaum2009modern}。
  \item 分时系统解决的主要问题是多用户使用分离的终端,却操作同一台计算机。
\end{enumerate}

% -------------

\subsection{第四代:个人计算机}

大规模集成电路进一步减小了计算机的大小,进一步为计算机进入大众视野做好了铺垫。
现在使用的台式机和笔记本就是属于第四代计算机发展的较高版本。

第四代计算机和之前的计算机比较而言在于同一台终端的使用人数,
第四代之前的计算机主要用于科研和大规模计算,所有硬件都是为这些作业而工作,操作系统也只用为这些作业安排调度,
而每个人都能拥有自己的计算机,那么计算机就需要有更多个性化的东西,这对操作系统提出了更多的要求:

\begin{enumerate}
  \item 更多的进程
  \item 更大的存储空间(文件管理)
  \item 更多设备接入(如鼠标)
  \item 网络要求的实现
  \item 对外防护的要求
  \item 对外提供接口(图形化显示)
\end{enumerate}

% -------------

\subsection{第五代:移动计算机}

第五代计算机是便携式计算机,也就是智能手机或者智能平板设备,较之第四代计算机最大的不同在于易于携带,
但是缺点也很显著,受到设备体积和电池的限制,性能受到很大的影响。

第五代计算机对操作系统有了新的要求:
\begin{enumerate}
  \item 从待机状态极快的响应
  \item 电量控制(即低成本完成作业)
\end{enumerate}

% ----------

\subsection{小结}

纵观操作系统的发展史,发展的中心问题是“如何更低成本完成更多的任务”,
而发展的关键节点是多道程序设计以及分时系统。

展望计算机的发展史,大胆的推测下一代计算机可能就是"computer everywhwere",
而那时操作系统也会有新的发展。

\section{操作系统的规范化}

\subsection{Unix及类Unix}
Unix 及类 Unix 将用户空间与系统空间划分开,以此规定内核的边界,将存在于系统空间的代码与数据的集合称为内核。
由此也存在了不同的 CPU 运行模式:系统态和用户态。
\begin{enumerate}
  \item 系统调用接口
  \item 进程管理
  \item 内存管理
  \item 虚拟文件系统
  \item 网络堆栈
  \item 设备驱动程序
\end{enumerate}

\subsection{Windows}

Microsoft Windows 与 Unix 最大的不同是,它将较低层的离硬件最近的一部分叫做“内核”,
因此,Windows也将图形界面和视窗机制的实现也放在了内核中。

大体来说,Windows内核层次如下:
\begin{enumerate}
  \item 系统调用接口
  \item 中断/异常入口
  \item Executive (管理层)、对象管理、内存管理、进程管理、安全管理、I/O管理等~\cite{毛德操2005windows}
  \item 核心层、设备驱动底层
  \item HAL(硬件抽象层)
\end{enumerate}

\subsection{小结}
从市场中用户数量较多的两大操作系统平台看,操作系统已经趋于规范化,功能方面也趋于成熟,
两大操作系统的功能都非常完备,但功能多了的坏处就是bug出现的概率也随之变高,
所以windows的蓝屏和linux的系统抖动等操作系统问题都对用户造成了极大的困扰。

想要解决这些问题较好的办法就是精简功能,从降低代码数量开始降低系统bug的出现概率,
精简功能则要很好的处理好必要功能和非必要功能的筛选分类。

% 启动 v
\section{启动空白操作系统}

\subsection{操作系统启动流程}

按下电源键后计算机开始启动,启动过程分为3个阶段~\cite{阮一峰2014如何变得有思想}:
\begin{center}BIOS -> MBR -> 操作系统\end{center}

\begin{enumerate}
\item 在BIOS完成POST(硬件自检,Power-On Self Test)
  并根据启动顺序(Boot Sequence)来选择启动设备。本系统是从U盘启动。
\item 计算机读取该设备的MBR(Master boot record,位于第一个扇区,即最前面的512个字节,见图~\ref{fig:mbr}),
  并运行其中的启动程序IPL(Initial Program Loader),将ZOS加载入内存。
  本部分的实现代码,参见附录程序~\ref{sec:ipl09}。
\item 控制权转交给操作系统后,Kernel开始运行,操作系统启动完成。
\end{enumerate}

\subsection{制作MBR}

MBR的主要部分是Bootstrap code area,即前446个字节。MBR结构见图~\ref{fig:mbr}。
负责指出操作系统的位置,主分区第一个扇区的物理位置(柱面、磁头、扇区号等等)
,参见附录程序\ref{sec:fat12}。

\begin{figure}[H]
  \centering
  \includegraphics[width=1\textwidth]{fig/mbr.pdf}
  \caption{MBR}
  \label{fig:mbr}
\end{figure}

一个扇区大小为512字节,MBR位于C0-H0-S1(柱面0,磁头0,扇区1),\cite{刘伟2010数据恢复技术深度揭秘}
从下一个扇区(扇区2,C0-H0-S2)开始加载操作系统。找到扇区2的操作如程序~\ref{lst:chs}所示。

\begin{listing}[H]
  \begin{codeblock}[.5]
  \inputminted[tabsize=2, firstline=43, lastline=45,
  linenos=true]{nasm}{../ZOS/src/kernel/ipl09.nas}
  \end{codeblock}
  \caption{初始化读取柱面、磁头和扇区的起点}
  \label{lst:chs}
\end{listing}

完成定位后开始将磁盘数据读入内存,策略是
\begin{enumerate}
\item 磁头0,柱面0,读取1-18扇区 (C0-H0-S2)-(C0,H0,S18)
\item 磁头1,柱面0,读取1-18扇区 (C0-H1-S1)-(C0-H1-S18)
\item 磁头0,柱面1,读取1-18扇区 (C1-H0-S1)-(C1-H0-S18)
\item ...
\item 磁头1,柱面78,读取1-18扇区 (C78-H1-S1)-(C78-H1-S18)
\item 磁头0,柱面79,读取1-18扇区 (C79-H0-S1)-(C79-H0-S18)
\item 磁头1,柱面79,读取1-18扇区 (C79-H1-S1)-(C79-H1-S18)
\end{enumerate}

按以上策略将磁盘内容读入内存,核心代码如附录程序~\ref{sec:readfrag}。

readfast 位于代码第76行,JMP readfast 在此代表循环执行。
当循环执行完毕,表示磁盘内容已经全部加载到内存中,
MBR过程成功,开始启动操作系统。

\subsection{制作空白操作系统}

为测试操作系统是否成功被MBR启动,设计将操作系统设置为启动后待机。
如程序~\ref{lst:blankos}所示。按下电源键,经过启动步骤系统循环执行
HLT\footnote{HLT: 让CPU停止动作并进入待机状态\cite{30_osKawaiHidemi200630}}
使得操作系统计算机始终处于待机状态,启动成功。

\begin{listing}[H]
  \begin{codeblock}[.5]
    \begin{nasmcode}
      fin:
      HLT
      JMP fin
    \end{nasmcode}
  \end{codeblock}
  \caption{空白操作系统}
  \label{lst:blankos}
\end{listing}

\section{操作系统组件功能全览}
在完成空白操作系统后,操作系统的设计才正式拉开序幕,
此次为操作系统设计的功能如图~\ref{fig:run}所示,
操作系统启动后将完成各种功能需求的初始化操作,
在初始化成功后各项功能就可以正常运行。

\begin{figure}[H]
  \centering
  \includegraphics[width=.8\textwidth]{fig/func/run.pdf}
  \caption{操作系统组件功能全览}
  \label{fig:run}
\end{figure}

\begin{description}
  \item[初始化] 初始化过程主要涉及到:\\
  内存的损坏检测,数据情况,系统数据写入;\\
  输入设备中键盘及鼠标的启用等;
  \item[内存管理] 内存管理包括内存的分配和释放;
  \item[输入输出] 输入输出包括键盘输入,鼠标输入,标准输出;
  \item[多道程序] 多道程序中设计多道程序设计,分时系统,程序切换及程序的优先级设置。
\end{description}

% 内核 
\chapter{编写操作系统内核}

从内存管理,输入输出,多进程,分时四个模块丰富操作系统的内容

\section{内存管理}

内存 (RAM) 是计算机中不可缺少的重要硬件,所有程序的运行都是在内存中进行的,
而CPU访问硬盘数据也必须先经过内存交换才得以实现,内存在加速CPU访问硬盘居功至伟。
由内存的重要性可知内存管理在操作系统中也非常重要。	

内存的结构通常用地址和空间表示,在计算机运行过程中,
内存的分配是随机的,导致内存的释放也是相对无序的,这样导致了很多的碎片化问题,
也就是随计算机运行时间变长,内存中到处遍布小块零散空闲空间,虽然零散空间总数很大,
但很难满足新进程序的内存需求,于是内存管理显得十分必要。

内存管理设计的主要目的是快速并且高效的分配内存空间,并在适当的时间释放并回收内存空间。
根据内存管理的设计目的。

\begin{listing}[H]
  \inputminted[tabsize=2, firstline=137, lastline=143,
    linenos=true]{c}{../ZOS/src/kernel/bootpack.h}
  \caption{数据结构-内存管理}
  \label{lst:mem}
\end{listing}

\begin{description}
\item[frees:]当前可用内存组数
\item[maxfrees:]可用内存组数的最大值
\item[lostsize:]释放失败的内存的大小总和
\item[losts:]释放失败次数
\end{description}

经过内存初始化和释放所有内存空间后,内存管理正常运行。

\subsection{内存分配}

内存的分配方式涉及到内存释放,好的分配方式会使得内存使用的效率大大提高。
根据内存的大小来划分内存如何使用,预计使用32KB用于内存分配的管理空间,
则共有4000组左右的内存用于分配给各个程序使用,每个组4KB。

每一组内存经过初始化都拥有自己的数据结构,
即每一组空闲内存的地址和大小都被记录到空闲内存表free。

\begin{table}[h]
  \centering
  \begin{tabular}{|c|c|}
    \hline 组号 & 地址 \\
    \hline 2 & 0x00005000 \\ 
    \hline 1 & 0x00004000 \\
    \hline 0 & 0x00003000 \\
    \hline
  \end{tabular}
  \caption{空闲内存表free}
  \label{tab:free}
\end{table}

一旦系统接收到程序申请内存的请求(需求的内存大小),
就开始在内存中寻找足够大的内存完成这次申请,并返回可供使用的空闲内存的地址。
完成申请后系统需要重新整理空闲内存表free,将可用内存组数减一,
将返回给程序空闲空间大小根据程序需求进行调整,并对剩余的可用内存表进行按地址升序整理。
流程图见\ref{fig:memman},程序见~\ref{lst:alloc}。

% ----------------------------

\subsection{内存释放}

为保证磁盘空闲空间尽可能少的碎片化,内存释放首先考虑的是使待释放空间与附近空闲空间进行合并\cite{bryant2003computer}。

具体分为三种情况:

\begin{description}
\item[前端空闲:]释放内存的相连前端是空闲内存或释放内存相连两端都是空闲内存
\item[后端可用:]释放内存的相连后端是空闲空间
\item[前端后端均不可用:]在当前位置释放内存
\end{description}

已知:待释放的空间的地址和空间大小

根据空闲内存表free的编号从0到frees遍历查找地址大于待释放空间的空闲内存,
并根据得到的空闲内存编号i及大小size区分此时的待释放内存应当采取何种方式释放,程序参见~\ref{lst:rw}。

前端空闲:

当相连前端有可用内存时将可释放内存大小归入前端可用内存内,frees不变;

当相连后端也有可用内存时将后端内存大小归入前端可用内存内,frees减一。

内存释放前后情况如图~\ref{fig:mem0}和图~\ref{fig:mem1}所示,程序见~\ref{lst:mem1}: 

\begin{figure}[h]
  \centering
  \includegraphics[width=.7\textwidth]{fig/mem0.pdf}
  \caption{前端空闲}
  \label{fig:mem0}
\end{figure}

\begin{figure}[h]
  \centering
  \includegraphics[width=.7\textwidth]{fig/mem1.pdf}
  \caption{前端可用,且后端空闲}
  \label{fig:mem1}
\end{figure}

% ------------------------------

\newpage
后端空闲:

当相连后端有可用内存的时候将free[i]的地址换为待释放内存的地址,相连后端内存大小归入待释放内存大小,frees不变。

内存释放前后情况如图~\ref{fig:mem2}所示,程序见~\ref{lst:mem2}。

\begin{figure}[h]
  \centering
  \includegraphics[width=.7\textwidth]{fig/mem2.pdf}
  \caption{后端空闲}
  \label{fig:mem2}
\end{figure}

% -----------------------------

前端后端均被占用:

由于被释放空间周围没有空闲内存,为保证free内各段内存仍然按照内存地址升序排列,
使空闲空间计数最大值加一,free[i]后续空闲内存序号加一,并将释放空间组号定为i。

内存释放前后情况如图~\ref{fig:mem3}所示,程序如~\ref{lst:mem3}。
\begin{figure}[h]
  \centering
  \includegraphics[width=.7\textwidth]{fig/mem3.pdf}
  \caption{前端后端均被占用}
  \label{fig:mem3}
\end{figure}

\section{输入输出}

输入作为人与计算机之间最基本的交互方式,其中键盘和鼠标是标准输入设备。

CPU通过PIC\footnote{PIC通过中断控制请求使用CPU,即在需要的时候对CPU发出中断请求,暂停CPU现在运行的作业,先完成中断请求再继续之前的作业}
(Programmable interrupt controller)与外部设备进行数据交换。
键盘作为计算机最早的外设,位于主PIC的IRQ1;
而鼠标作为第四代计算机才出现的输入设备,位于从PIC的IRQ12,而从PIC连接在主PIC的IRQ2。

输入设备输入时是将一个个指令\footnote{每一次敲击键盘按键,移动鼠标,点击鼠标都会产生响应的指令}
发送到CPU,而CPU同时只能处理一个指令(虽然处理时间非常短),
所以需要一个缓冲区fifo\footnote{缓冲区fifo采用先入先出的数据结构,使得先传入的输入指令能够率先得到执行}来接收输入数据。

\begin{listing}[H]
  \inputminted[tabsize=2, firstline=40, lastline=44,
    linenos=true]{c}{../ZOS/src/kernel/bootpack.h}
  \caption{数据结构-缓冲区fifo}
  \label{lst:fifo}
\end{listing}
\begin{description}
\item[*buf:]缓冲区在内存中的地址;
\item[p, q:]下一个写入地址,下一个数据读入地址;
\item[size, free:]缓冲区大小,空闲空间大小;
\item[TASK flags:]溢出标记(-1和0分别表示有溢出和无溢出);
\item[TASK *task:]在当前位置释放内存。
\end{description}

操作系统运行后始终通过函数fifo32\_status(\&fifo)\footnote{存储总量=size-free}
监测fifo缓冲区内是否有数据,如果有数据才执行下一步对输入指令的响应。
接收输入信号的函数fifo32\_get(\&fifo)负责从缓冲区fifo中取出一个数据(键盘为1个字节,鼠标为4个字节)交由CPU处理。
流程图见\ref{fig:fifo},程序见\ref{lst:kbd}.

计算机收到数据后根据数据的大小区间区分这一段数据是何处传来的并进行相应处理。
一旦收到的指令为256到511,系统判定为键盘输入;
当指令为512到767,系统判定为鼠标输入。

% --------------------

\subsection{键盘输入}

键盘作为最基础也是大众使用最精确的输入设备,它负担着很多的责任:
\begin{enumerate}
\item 输入文本按键
\item 修改文本输入的基本指令按键
\item 修改计算机的状态特殊按键
\end{enumerate}

按下不同的功能按键,键盘向计算机发送的指令是不一样的。

若输入指令为256-0x80+256时,为输入文本指令;

输入指令为退格键,回车键;

修改计算机的状态特殊按键如下:
\begin{table}[!ht]
  \centering
  \begin{tabular}{|l|c|l|c|}
    \hline 功能 & 指令 & 功能 & 指令 \\
    \hline 左Shift & 256+0x2a & CapsLock & 256+0x3a \\ 
    \hline 右Shift & 256+0x36 & NumLock off & 256+0xaa \\
    \hline 左Shift off & 256+0xaa & ScrollLock & 256+0xaa \\
    \hline 右Shift off & 256+0xb6 & & \\
    \hline
  \end{tabular}
  \caption{修改状态的特殊指令}
  \label{tab:hello}
\end{table}

% ---------------------

\subsection{鼠标输入}

鼠标作为第四代计算机才出现的输入设备,因为涉及到坐标位置的变化,它的输入指令较键盘也相对复杂,
从鼠标传到计算机的信号都是4个字节一组的,其中第一个字节表示状态,其他3个有效指令分别为x坐标,y坐标以及鼠标按键状态。

\begin{listing}[H]
  \inputminted[tabsize=2, firstline=126, lastline=129,
    linenos=true]{c}{../ZOS/src/kernel/bootpack.h}
  \caption{数据结构-鼠标输入的数据}
  \label{lst:mouse_data}
\end{listing}
\begin{description}
\item[buf:]存放鼠标一个数据指令的数组;
\item[phase:]当前鼠标工作的阶段;
\item[x, y]缓冲区大小,空闲空间大小;
\item[btn]按键状态。
\end{description}

% ----------------------

\subsection{标准输出}

由于是面对硬件开发操作系统,无法使用各种成熟的库和函数,
所以输出也只能从修改VRAM的一个个像素开始。

流程如下:
通过一系列指令后,系统需要向屏幕打印字符,
则从数组中读取一个数据,并从字体库(包含基本字体的像素化数组)中找到对应的显示方法,
写入VRAM(Video RAM),在屏幕对应坐标范围内显示数据,
循环读取下一个并显示下一个数据。

\section{多道程序系统与分时操作系统}

\subsection{多道程序系统}

现代的计算机已经不仅仅作为数字计算的工具,进入大众生活的计算机被赋予了更多生活上的需求,
用户可能在看电影的同时查看电子邮件,也有可能在写论文的时候进入浏览器查询相关资料,
但是更重要的是计算机往往在用户不经意间打开防病毒软件等保证用户计算机的安全。

由此可见多进程的工作方式在计算机工作中不可缺少。

但是在实际的处理过程中,计算机并不能同时处理多个程序。

首先要处理的问题是如何运行多个程序,早期的多道程序设计的出发点是充分的利用CPU,
作为输入设备的打孔纸带与CPU速度相比差距过大,昂贵的CPU常常在等待I/O信号而闲置。

具体情形如:A作业等待磁盘或者其他I/O时,CPU暂停为A作业服务,而转向为已完成I/O操作的B作业服务,
等到B需要执行下一步I/O操作时,CPU发现A作业完成了I/O操作,又转向为A作业服务,依次循环直到队列中所有作业完成。

由于A作业和B作业并行存储在计算机内存中,虽然在具体的执行中多次交换先后顺序执行,
但两个作业输出在同一个磁盘,故认为是多道程序在计算机中运行。

\subsection{分时操作系统}

分时是使得在用户看来计算机的多道程序同时运行,但是同时运行这也是不可能的。

所以只能采取折中的办法,使得CPU在用户不能明显感觉到的时间间隔内切换运行多个程序,
在切换后每个程序都能对作业进行一定的处理,在进行多个周期后,各个程序先后完成作业。

“不能明显感觉到的时间间隔内切换”中有两个概念,时间间隔和切换:

\begin{enumerate}
\item 时间间隔太长则用户会有明显的卡顿感,不利于分时概念的实现,
  而时间间隔太短则时间不足以让程序响应并完成一定量的工作,同样不利于分时概念的实现。
\item 切换涉及到保存当前程序的运行状态以便于程序获得CPU时间后可以接续上次的任务继续执行。
\end{enumerate}

\subsection{定时器}

定时器的实现是实现分时操作系统的关键,定时器每隔一段时间就向CPU发送中断信号,并记录发送的次数,以定时器发送的中断次数作为定时的依据。

管理定时器主要是使用PIT\footnote{PIT连接IRQ0,设定PIT就可以设定IRQ0的中断间隔,中断频率=单位时间事钟周期书(主频)/设定的数值}
(Programmable Interval Timer),之前说到定时器时间太短不利于分时的实现,故暂定1秒中发生100次中断。

在实际的运用中,往往需要用到不止一个定时器,于是需要对多个定时器进行管理,数据结构如下:

\begin{listing}[H]
  \inputminted[tabsize=2, firstline=175, lastline=187,
    linenos=true]{c}{../ZOS/src/kernel/bootpack.h}
  \caption{数据结构-多定时器}
  \label{lst:multi_timer}
\end{listing}

\begin{description}
\item[Timer *next:]下一个定时器地址;
\item[timeout:]下一个超时时刻;
\item[flags, flags2:]各个定时器的状态,在应用程序结束时定时器是否取消的标记
\item[count, next:]当前时刻,下一个时刻;
\item[Timer *t0:]所有时刻都要减去这个值。
\end{description}

对每个定时任务设置到超时时刻(对每个任务的运行时间进行计量,超过指定时刻向系统发送指令),
这样可以对一些设定了定时的任务在时间到了之后执行指定操作,
如光标的闪烁,页面的刷新等。

\subsection{任务切换}

在多道程序系统和分时操作系统分时操作系统完成后,任务切换就可以进行了。

任务切换从字面上理解很简单,多个任务之间来回切换,但是任务切换在计算机中实现却不那么简单。

如现有A和B两个任务,A正在运行,现在需要切换到B,流程如下:
\begin{enumerate}
\item 任务B向CPU发送切换任务的指令
\item CPU把当前寄存器中的值全部写到内存中
\item CPU执行任务B
\item CPU切换到任务A并把所有寄存器中的值从内存中读出来,继续执行任务A
\end{enumerate}
为了实现复杂的任务管理,需要用到的数据结构如下:
\begin{listing}[H]
  \inputminted[tabsize=2, firstline=227, lastline=232,
    linenos=true]{c}{../ZOS/src/kernel/bootpack.h}
  \inputminted[tabsize=2, firstline=222, lastline=226,
    linenos=true]{c}{../ZOS/src/kernel/bootpack.h}
  \inputminted[tabsize=2, firstline=209, lastline=221,
    linenos=true]{c}{../ZOS/src/kernel/bootpack.h}
  \caption{数据结构-多任务}
  \label{lst:multi_task}
\end{listing}

\subsection{任务优先级}

在实际的运用中,所有任务在同一个优先级的安排显然是不合理的,
比如系统在执行A作业时,由于A作业和键盘输入作业优先级相同,
所以系统可能认为应该先完成A作业再执行键盘输入。

而通常人为操作中键盘优先级应该是最高的,为此设计任务的优先级。

\begin{listing}[H]
  \inputminted[tabsize=2, firstline=222, lastline=226,
    linenos=true]{c}{../ZOS/src/kernel/bootpack.h}
  \caption{数据结构-任务优先级}
  \label{lst:task_level}
\end{listing}

设计中,将任务分为多个层次,将音乐、网络传输等对优先级要求高的任务放在最高的优先级,
将鼠标等任务放在稍低的优先级,将对优先级要求低的任务放在较低的优先级。

则系统在多任务同时运行的情况下会优先处理高优先级的任务,处理完成后处理稍低的优先级任务,
优先级较低的任务将在系统较空闲(即无更高优先级任务)的时候处理。

% % memory
% \subsection{内存分配}

内存的分配方式与之后的内存释放息息相关,好的分配方式会使得效率大大提高。
根据内存的大小来划分内存如何使用,预计使用32KB用于内存分配的管理空间,
则共有4000组左右的内存用于分配给各个程序使用。

每一组内存经过初始化都拥有自己的数据结构,
即每一组空闲内存的地址和大小都被记录到空闲内存表free。

\begin{table}[!ht]
  \centering
  \begin{tabular}{|c{4cm}|c{8cm}|}
    \hline 
    组号 & 地址 \\
    \hline 
    1 & 0x00003000 \\ 
    \hline 
    2 & 0x00004000 \\
    \hline 
    3 & 0x00005000 \\
    \hline
  \end{tabular}
  \caption{表格示例}
  \label{tab:hello}
\end{table}

一旦系统接收到程序申请内存的请求(需求的内存大小),
就开始在内存中寻找足够大的内存完成这次申请,并返回可供使用的空闲内存的地址。
完成申请后系统需要重新整理空闲内存表free,将最大可用内存组数减一,
将返回给程序空闲空间大小根据程序需求进行调整,并对剩余的可用内存表进行整理。

\begin{listing}[H]
  \inputminted[tabsize=2, firstline=68, lastline=78,
  linenos=true]{c}{../ZOS/src/kernel/memory.c}
\end{listing}

\newpage
\subsection{内存释放}

为保证磁盘空闲空间尽可能少的碎片化,内存释放首先考虑的是使待释放空间与附近空闲空间进行合并。

具体分为三种情况:

\begin{description}
\item[前端空闲:]释放内存的相连前端是空闲内存或释放内存相连两端都是空闲内存
\item[后端可用:]释放内存的相连后端是空闲空间
\item[前端后端均不可用:]挪动空闲空间以合并
\end{description}

已知:待释放的空间的地址和空间大小

根据空闲内存表free的编号查找地址大于待释放空间的空闲内存,
并根据得到的空闲空间编号及大小区分此时的待释放内存应当采取何种方式释放。
\begin{listing}[H]
  \inputminted[tabsize=2, firstline=91, lastline=95,
  linenos=true]{c}{../ZOS/src/kernel/memory.c}
\end{listing}

\newpage
前端空闲:

当待释放的空间前方有空闲空间时,将free[i-1]的大小加上释放空间的大小

内存释放前后情况如图\ref{fig:mem0}和图\ref{fig:mem1}所示: 

\begin{figure}[h]
  \centering
  \includegraphics[width=.8\textwidth]{fig/mem0.pdf}
  \caption{前端空闲}
  \label{fig:mem0}
\end{figure}

当待释放的空间后方有空闲空间时,将free[i-1]的大小加上释放空间的大小
\begin{figure}[h]
  \centering
  \includegraphics[width=.8\textwidth]{fig/mem1.pdf}
  \caption{前端可用,且后端空闲}
  \label{fig:mem1}
\end{figure}

\newpage
实现如下:

\begin{listing}[H]
  \inputminted[tabsize=2, firstline=98, lastline=116,
  linenos=true]{c}{../ZOS/src/kernel/memory.c}
\end{listing}

\newpage
后端空闲:

内存释放前后情况如图\ref{fig:mem2}所示:

当待释放的空间后方有空闲空间时,将free[i-1]的大小加上释放空间的大小
\begin{figure}[h]
  \centering
  \includegraphics[width=.8\textwidth]{fig/mem2.pdf}
  \caption{后端空闲}
  \label{fig:mem2}
\end{figure}

实现如下:

\begin{listing}[H]
  \inputminted[tabsize=2, firstline=118, lastline=127,
  linenos=true]{c}{../ZOS/src/kernel/memory.c}
\end{listing}

\newpage
前端后端均被占用:

内存释放前后情况如图\ref{fig:mem3}所示:

由于被释放空间周围没有空闲内存,为保证free内各段内存仍然按照内存地址升序排列,
使空闲空间计数最大值加一,free[i]后续空闲内存序号加一,并将释放空间序号定为i。
\begin{figure}[h]
  \centering
  \includegraphics[width=.8\textwidth]{fig/mem3.pdf}
  \caption{前端后端均被占用}
  \label{fig:mem3}
\end{figure}

实现如下:
\begin{listing}[H]
  \inputminted[tabsize=2, firstline=128, lastline=141,
  linenos=true]{c}{../ZOS/src/kernel/memory.c}
\end{listing}
% % io
% \documentclass{standalone}
\usepackage{xeCJK}
\usepackage{tikz}

\usetikzlibrary{shapes.geometric, arrows.meta, arrows,positioning}

\tikzstyle{block}=[rectangle, draw=black, thick,
    text width=10em,align=center, rounded corners,
    minimum height=4em]

\tikzstyle{blank}=[rectangle, thick,
    text width=10em,align=center, rounded corners,
    minimum height=4em]

\tikzstyle{block_left}=[rectangle, draw=black, thick, fill=white,
    text width=12em, text ragged, minimum height=4em, inner sep=6pt]

\tikzstyle{line}=[draw, thick, shorten >=2pt, ->]

\tikzstyle{arrow} = [->,>=stealth]

\newcommand*{\h}{\hspace{5pt}}% for indentation
\newcommand*{\hh}{\h\h}% double indentation
  
% \usetikzlibrary{shapes.geometric, arrows,positioning,calc}
% \usetikzlibrary{arrows,decorations.pathmorphing,backgrounds,positioning,fit,petri}
% \usetikzlibrary{shapes,arrows,intersections,patterns}

% \tikzstyle{startstop} = [rectangle, rounded corners, minimum width=2cm, minimum
% height=0.5cm, align=center, draw]

% \tikzstyle{io} = [trapezium, trapezium left angle=70, trapezium right angle=110, align=center, draw]

% \tikzstyle{process} = [rectangle, align=center, draw]

% \tikzstyle{oval}=[ellipse, align=center, draw]

% \tikzstyle{decision} = [shape aspect=2,diamond, align=center, draw]

% \tikzstyle{arrow} = [thick,->,>=stealth]
% \tikzstyle{dotArrow} = [dotted, ->, >=stealth]
% \tikzstyle{line} = [-]



    
\begin{document}
  \begin{tikzpicture}
  
    \matrix [column sep=5mm,row sep=7mm]
    {
        \node [block] (start) {start}; \\

        \node [block] (init_kbd)[xshift=-4cm]{init\_keyboard()}; & 
        \node [block] (eab_ms)[xshift=1cm]{enable\_mouse()}; \\

        \node [block] (fifo32_status) [xshift=-4cm]{fifo32\_status()}; & 
        \node [block] (fifo32_get) [xshift=1cm]{fifo32\_get()}; \\
    };
    \begin{scope}[every path/.style=line]
        \path (start) -- (init_kbd);
        \path (start) -- (eab_ms);
        \path (fifo32_status) -- (fifo32_get);
        \path (init_kbd) -- (fifo32_get);
        \path (eab_ms) -- (fifo32_get);
    \end{scope}
    \end{tikzpicture}
\end{document}
  
  %%% Local Variables:
  %%% mode: latex
  %%% TeX-master: t
  %%% End:
  
% % multi
% \section{多道程序系统}

现代的计算机已经不仅仅作为数字计算的工具,而进入大众生活的计算机被赋予了更多的生活需求,
用户可能在看电影的同时查看电子邮件,也有可能在写论文的时候进入浏览器查询相关资料,
但是更重要的是计算机往往在用户不经意间打开防病毒软件等保证用户计算机的安全。

由此可见多进程的工作方式在计算机工作中不可缺少。

但是在实际的处理过程中,计算机并不能同时处理多个程序,所以必须采用分时的设计,
关于分时操作系统的设计在下一节。
在此有两个概念,同时处理和多个程序,同时处理属于分时,多道程序属于多道程序设计。

首先要处理的问题是如何运行多个程序,早期的多道程序设计的出发点是充分的利用CPU,
作为输入设备的打孔纸带与CPU速度相比差距过大,昂贵的CPU常常在等待I/O信号而闲置。

具体情形如:A作业等待磁盘或者其他I/O时,CPU暂停为A作业服务,而转向为已I/O操作已完成的B作业服务。

由于A作业和B作业并行存储在计算机内存中,操作员无法区分现在是正在运行的是A作业还是B作业,
但两个作业输出在同一个磁盘,故认为是多道程序在计算机中运行。



  % memory v
  \subsection{内存分配}

内存的分配方式与之后的内存释放息息相关,好的分配方式会使得效率大大提高。
根据内存的大小来划分内存如何使用,预计使用32KB用于内存分配的管理空间,
则共有4000组左右的内存用于分配给各个程序使用。

每一组内存经过初始化都拥有自己的数据结构,
即每一组空闲内存的地址和大小都被记录到空闲内存表free。

\begin{table}[!ht]
  \centering
  \begin{tabular}{|c{4cm}|c{8cm}|}
    \hline 
    组号 & 地址 \\
    \hline 
    1 & 0x00003000 \\ 
    \hline 
    2 & 0x00004000 \\
    \hline 
    3 & 0x00005000 \\
    \hline
  \end{tabular}
  \caption{表格示例}
  \label{tab:hello}
\end{table}

一旦系统接收到程序申请内存的请求(需求的内存大小),
就开始在内存中寻找足够大的内存完成这次申请,并返回可供使用的空闲内存的地址。
完成申请后系统需要重新整理空闲内存表free,将最大可用内存组数减一,
将返回给程序空闲空间大小根据程序需求进行调整,并对剩余的可用内存表进行整理。

\begin{listing}[H]
  \inputminted[tabsize=2, firstline=68, lastline=78,
  linenos=true]{c}{../ZOS/src/kernel/memory.c}
\end{listing}

\newpage
\subsection{内存释放}

为保证磁盘空闲空间尽可能少的碎片化,内存释放首先考虑的是使待释放空间与附近空闲空间进行合并。

具体分为三种情况:

\begin{description}
\item[前端空闲:]释放内存的相连前端是空闲内存或释放内存相连两端都是空闲内存
\item[后端可用:]释放内存的相连后端是空闲空间
\item[前端后端均不可用:]挪动空闲空间以合并
\end{description}

已知:待释放的空间的地址和空间大小

根据空闲内存表free的编号查找地址大于待释放空间的空闲内存,
并根据得到的空闲空间编号及大小区分此时的待释放内存应当采取何种方式释放。
\begin{listing}[H]
  \inputminted[tabsize=2, firstline=91, lastline=95,
  linenos=true]{c}{../ZOS/src/kernel/memory.c}
\end{listing}

\newpage
前端空闲:

当待释放的空间前方有空闲空间时,将free[i-1]的大小加上释放空间的大小

内存释放前后情况如图\ref{fig:mem0}和图\ref{fig:mem1}所示: 

\begin{figure}[h]
  \centering
  \includegraphics[width=.8\textwidth]{fig/mem0.pdf}
  \caption{前端空闲}
  \label{fig:mem0}
\end{figure}

当待释放的空间后方有空闲空间时,将free[i-1]的大小加上释放空间的大小
\begin{figure}[h]
  \centering
  \includegraphics[width=.8\textwidth]{fig/mem1.pdf}
  \caption{前端可用,且后端空闲}
  \label{fig:mem1}
\end{figure}

\newpage
实现如下:

\begin{listing}[H]
  \inputminted[tabsize=2, firstline=98, lastline=116,
  linenos=true]{c}{../ZOS/src/kernel/memory.c}
\end{listing}

\newpage
后端空闲:

内存释放前后情况如图\ref{fig:mem2}所示:

当待释放的空间后方有空闲空间时,将free[i-1]的大小加上释放空间的大小
\begin{figure}[h]
  \centering
  \includegraphics[width=.8\textwidth]{fig/mem2.pdf}
  \caption{后端空闲}
  \label{fig:mem2}
\end{figure}

实现如下:

\begin{listing}[H]
  \inputminted[tabsize=2, firstline=118, lastline=127,
  linenos=true]{c}{../ZOS/src/kernel/memory.c}
\end{listing}

\newpage
前端后端均被占用:

内存释放前后情况如图\ref{fig:mem3}所示:

由于被释放空间周围没有空闲内存,为保证free内各段内存仍然按照内存地址升序排列,
使空闲空间计数最大值加一,free[i]后续空闲内存序号加一,并将释放空间序号定为i。
\begin{figure}[h]
  \centering
  \includegraphics[width=.8\textwidth]{fig/mem3.pdf}
  \caption{前端后端均被占用}
  \label{fig:mem3}
\end{figure}

实现如下:
\begin{listing}[H]
  \inputminted[tabsize=2, firstline=128, lastline=141,
  linenos=true]{c}{../ZOS/src/kernel/memory.c}
\end{listing}
  % io 键盘v-鼠标-输出
  \documentclass{standalone}
\usepackage{xeCJK}
\usepackage{tikz}

\usetikzlibrary{shapes.geometric, arrows.meta, arrows,positioning}

\tikzstyle{block}=[rectangle, draw=black, thick,
    text width=10em,align=center, rounded corners,
    minimum height=4em]

\tikzstyle{blank}=[rectangle, thick,
    text width=10em,align=center, rounded corners,
    minimum height=4em]

\tikzstyle{block_left}=[rectangle, draw=black, thick, fill=white,
    text width=12em, text ragged, minimum height=4em, inner sep=6pt]

\tikzstyle{line}=[draw, thick, shorten >=2pt, ->]

\tikzstyle{arrow} = [->,>=stealth]

\newcommand*{\h}{\hspace{5pt}}% for indentation
\newcommand*{\hh}{\h\h}% double indentation
  
% \usetikzlibrary{shapes.geometric, arrows,positioning,calc}
% \usetikzlibrary{arrows,decorations.pathmorphing,backgrounds,positioning,fit,petri}
% \usetikzlibrary{shapes,arrows,intersections,patterns}

% \tikzstyle{startstop} = [rectangle, rounded corners, minimum width=2cm, minimum
% height=0.5cm, align=center, draw]

% \tikzstyle{io} = [trapezium, trapezium left angle=70, trapezium right angle=110, align=center, draw]

% \tikzstyle{process} = [rectangle, align=center, draw]

% \tikzstyle{oval}=[ellipse, align=center, draw]

% \tikzstyle{decision} = [shape aspect=2,diamond, align=center, draw]

% \tikzstyle{arrow} = [thick,->,>=stealth]
% \tikzstyle{dotArrow} = [dotted, ->, >=stealth]
% \tikzstyle{line} = [-]



    
\begin{document}
  \begin{tikzpicture}
  
    \matrix [column sep=5mm,row sep=7mm]
    {
        \node [block] (start) {start}; \\

        \node [block] (init_kbd)[xshift=-4cm]{init\_keyboard()}; & 
        \node [block] (eab_ms)[xshift=1cm]{enable\_mouse()}; \\

        \node [block] (fifo32_status) [xshift=-4cm]{fifo32\_status()}; & 
        \node [block] (fifo32_get) [xshift=1cm]{fifo32\_get()}; \\
    };
    \begin{scope}[every path/.style=line]
        \path (start) -- (init_kbd);
        \path (start) -- (eab_ms);
        \path (fifo32_status) -- (fifo32_get);
        \path (init_kbd) -- (fifo32_get);
        \path (eab_ms) -- (fifo32_get);
    \end{scope}
    \end{tikzpicture}
\end{document}
  
  %%% Local Variables:
  %%% mode: latex
  %%% TeX-master: t
  %%% End:
  
  % 多道程序设计
  \section{多道程序系统}

现代的计算机已经不仅仅作为数字计算的工具,而进入大众生活的计算机被赋予了更多的生活需求,
用户可能在看电影的同时查看电子邮件,也有可能在写论文的时候进入浏览器查询相关资料,
但是更重要的是计算机往往在用户不经意间打开防病毒软件等保证用户计算机的安全。

由此可见多进程的工作方式在计算机工作中不可缺少。

但是在实际的处理过程中,计算机并不能同时处理多个程序,所以必须采用分时的设计,
关于分时操作系统的设计在下一节。
在此有两个概念,同时处理和多个程序,同时处理属于分时,多道程序属于多道程序设计。

首先要处理的问题是如何运行多个程序,早期的多道程序设计的出发点是充分的利用CPU,
作为输入设备的打孔纸带与CPU速度相比差距过大,昂贵的CPU常常在等待I/O信号而闲置。

具体情形如:A作业等待磁盘或者其他I/O时,CPU暂停为A作业服务,而转向为已I/O操作已完成的B作业服务。

由于A作业和B作业并行存储在计算机内存中,操作员无法区分现在是正在运行的是A作业还是B作业,
但两个作业输出在同一个磁盘,故认为是多道程序在计算机中运行。


  % ctss
  \section{分时操作系统}

在上一节中说到分时是使得在用户看来计算机的多道程序同时运行,多道程序已经实现了,
分时简单说是使得CPU在用户不能明显感觉到的时间间隔内切换运行多个程序,
在切换后每个程序都能对作业进行一定的处理,在进行多个周期后,各个程序先后完成作业。

不能明显感觉到的时间间隔内切换中有两个概念,时间间隔和切换:

时间间隔太长则用户会有明显的卡顿感,不利于分时概念的实现,
而间隔时间太短则时间不足以让程序响应并完成一定量的工作,同样不利于分时概念的实现。

切换涉及到保存当前程序的运行状态以便于程序获得时间片后可以接续上次的任务继续执行。



% 兼容 缺
\chapter{对外兼容及安全防护}

从接口设计及安全防护的角度完善操作系统

\section{系统调用}

操作系统是用于管理硬件的软件,但是对绝大部分人都不算友好,
可友好不是操作系统必须的,操作系统只需要快速和高效,
为此操作系统向外界开放接口,
由其他的编程人员来使用接口完成各种功能的实现,
比如:图形界面,各种字处理软件等。

系统调用的实现是通过程序向操作系统申请权限访问各种指定的系统函数,
操作系统操作硬件完成工作。

向外界开发的系统调用接口:
\begin{enumerate}
    \item 显示单个字符
    \item 显示字符串
    \item 键盘输入
    \item 定时器
    \item 文件操作
    \item 命令行
\end{enumerate}

利用这些接口可以实现程序如下:
\begin{enumerate}
    \item 命令行计算器(调用键盘输入、显示字符串、命令行)
    \item 文本阅读器(调用文件操作)
    \item 音乐播放器(调用命令行、文件操作、定时器)
    \item 图片阅读器(调用命令行、文件操作)
\end{enumerate}

\subsection{命令行计算器}

    \subsubsection{预期设计}
        使用方法设计:calc+计算公式。 \\
        结果设计:分别以十进制和十六进制表示。

    \subsubsection{流程设计}
    命令行计算器的实现流程如图~\ref{fig:calc}所示。
        \begin{figure}[H]
            \centering
            \includegraphics[width=\textwidth]{../Fig/api/calc.pdf}
            \caption{命令行计算器}
            \label{fig:calc}
        \end{figure}
        
        \csingle|int api_cmdline(char *buf, int maxsize);|
        \begin{itemize}
        \item 调用命令行系统接口
        \end{itemize}

        \csingle|int getnum(char **pp, int priority);|
        \begin{itemize}
        \item 获得数字与符号并计算
        \end{itemize}

        \csingle|void api_putstr0(char *s);|
        \begin{itemize}
        \item 输出运算结果
        \end{itemize}

        \csingle|void api_end(void);|
        \begin{itemize}
        \item 结束调用系统借口
        \end{itemize}

\subsection{文本阅读器}

    \subsubsection{预期设计}
        使用方法设计:tview+文本文件名。 \\
        结果设计:打开新窗口并显示文本内容。
    \subsubsection{实现过程}
    文本阅读器的实现流程如图~\ref{fig:tview}所示。
    % ../ZOS/src/apps/tview/tview.c

      \begin{figure}[H]
        \centering
        \includegraphics[width=\textwidth]{../Fig/api/tview.pdf}
        \caption{文本阅读器}
        \label{fig:tview}
      \end{figure}

    \csingle|api_fopen(char *fname);|
    \begin{itemize}
    \item 利用系统接口打开文件
    \end{itemize}
    
    \csingle|api_fsize(int fhandle, int mode);|
    \begin{itemize}
    \item 获取文件大小
    \end{itemize}

    \csingle|api_fread(char *buf, int maxsize, int fhandle);|
    \begin{itemize}
    \item 读取文件内容
    \end{itemize}

    \csingle|api_fclose(int fhandle);|
    \begin{itemize}
    \item 关闭文件
    \end{itemize}

    \csingle|textview(int win, int w, int h, int xskip, char *p, int tab, int lang)|
    \begin{itemize}
    \item 文件内容显示
    \end{itemize}

\subsection{音乐播放器}
    \subsubsection{预期设计}

    使用方法设计:mmlplay+歌曲文件名。

    结果设计:打开新窗口并播放音乐。
    \subsubsection{实现过程}
    音乐播放器的实现流程如图~\ref{fig:mmlplay}所示。
    \begin{figure}[H]
        \centering
        \includegraphics[width=\textwidth]{../Fig/api/mmlplay.pdf}
        \caption{音乐播放器}
        \label{fig:mmlplay}
      \end{figure}

    \csingle|api_alloctimer(void);|
    \begin{itemize}
    \item 分配定时器
    \end{itemize}

    \csingle|api_inittimer(int timer, int data);|
    \begin{itemize}
    \item 初始化定时器
    \end{itemize}

    \csingle|waittimer(int timer, int time);|
    \begin{itemize}
    \item 等待定时器
    \end{itemize}

    \csingle|api_beep(int tone);|
    \begin{itemize}
    \item 播放对应音符
    \end{itemize}

\subsection{图片阅读器}
    \subsubsection{预期设计}

    使用方法设计:gview+图片文件名。

    结果设计:打开新窗口并显示图片。
    \subsubsection{实现过程}
    图片阅读器的实现流程如图~\ref{fig:gview}所示。
    \begin{figure}[H]
        \centering
        \includegraphics[width=.7\textwidth]{../Fig/api/gview.pdf}
        \caption{图片阅读器}
        \label{fig:gview}
      \end{figure}

    \csingle|info_BMP(struct DLL_STRPICENV *env, int *info, int size, char *fp);|
    \begin{itemize}
    \item 检查BMP文件信息
    \end{itemize}

    \csingle|info_JPEG(struct DLL_STRPICENV *env, int *info, int size, char *fp);|
    \begin{itemize}
    \item 检查JPG文件信息
    \end{itemize}

    \csingle|decode0_BMP(struct DLL_STRPICENV *env, int size, char *fp, int b_type, char *buf, int skip);|
    \begin{itemize}
    \item 解码BMP文件
    \end{itemize}

    \csingle|decode0_JPEG(struct DLL_STRPICENV *env, int size, char *fp, int b_type, char *buf, int skip);|
    \begin{itemize}
    \item 解码JPEG文件
    \end{itemize}

    \csingle|rgb2pal(int r, int g, int b, int x, int y);|
    \begin{itemize}
    \item 播放对应音符
    \end{itemize}

\section{系统安全}

只有完全封闭的系统才有可能成为真正安全的操作系统,但是既然选择了对外开放接口,
就必须重视外部应用程序的各种有意或无意的操作给操作系统带来的安全隐患,
一旦有程序跨越操作系统直接操作硬件很可能使得操作系统崩溃。

操作系统在开放接口的同时也需要一定的安全举措:
\begin{enumerate}
    \item 内存写保护:若将内存直接暴露给应用程序使得应用程序可能不慎在错误的地方写入错误的值,
    那么导致的问题可能包括系统崩溃、文件损坏,甚至导致计算机硬件损坏
    \item 系统函数权限:管理员的权限是通过各个系统函数实现的,将系统函数开放给对系统不熟悉的用户是完全不负责任的)
    \item 监测并终端异常程序:计算机病毒是自发的运行并对计算机造成意想不到后果的恶意程序,
    所以在操作系统发现有程序活动异常时应该立即中断此程序
    \item 系统调用防护:系统调用的目的是使得外部开发者通过系统设计者的意愿得到希望得到的系统函数执行权而开发应用程序,
    于是在系统调用方面,每一个系统调用接口都应该绑定固定的系统函数,并限制其用法,拒绝非系统调用的方式启用系统函数。
\end{enumerate}

%%% 正文部分到此结束。下面是『参考文献』、『指导教师简介』、『鸣谢』、『附录』

%% 不要动下面四行!
\Appendix{}
\printbibliography[heading={bibintoc},title={参考文献}] % 输出参考文献
\advisorinfopage{}                 % 输出指导教师简介
\acknowledgmentspage{}             % 输出鸣谢

%%% 下面是附录部分,可以没有。

% 附录
\chapter{文件清单}

\section{根目录}
\begin{description}
    \item[includes/] 调用的C函数库
    \item[kernel/] 系统kernel目录
    \item[apps/] 系统调用演示程序
    \item[demos/] 系统测试程序
\end{description}

\section{Kernel目录}
\begin{description}
    \item[asmhead.nas]                                                                                                                                        
    \item[bootpack.c] 主程序                                                                                                                                      
    \item[bootpack.h] 头文件                                                                                                                                           
    \item[dsctbl.c] GDT,IDT控制程序                                                                                                                                          
    \item[fifo.c] 缓冲区                                                                                                                           
    \item[file.c] 文件管理                                                                                                                                            
    \item[graphic.c] 图像显示及控制                                                                                                                                     
    \item[int.c] 中断                                                                                                                                             
    \item[ipl09.nas] 启动程序IPL                                                                                                                                         
    \item[keyboard.c] 键盘控制                                                                                                                                       
    \item[memory.c] 内存管理                                                                                                                                          
    \item[mouse.c] 鼠标控制                                                                                                                                           
    \item[mtask.c] 多程序控制                                                                                                                                           
    \item[naskfunc.nas] 汇编函数库                                                                                                                                      
    \item[sheet.c] 图层控制                                                                                                                                           
    \item[timer.c] 定时器                                                                                                                                           
    \item[window.c] 窗口控制
\end{description}







% 程序
\chapter{主要程序代码} %附录二

\section{初始启动程序代码(ipl09.nas)节选}
\label{sec:ipl09.nas}
\begin{listing}[H]
  \inputminted[tabsize=2, firstline=12, lastline=29,
  linenos=true]{nasm}{../ZOS/src/kernel/ipl09.nas}
  \caption{FAT12格式磁盘专用代码}
  \label{sec:fat12}
\end{listing}

\begin{listing}[H]
  \inputminted[tabsize=2, firstline=76, lastline=88,
  linenos=true]{nasm}{../ZOS/src/kernel/ipl09.nas}
  \caption{将磁盘内容读入内存}
  \label{sec:read}
\end{listing}

\begin{listing}[H]
  \inputminted[tabsize=2, firstline=138, lastline=147,
  linenos=true]{nasm}{../ZOS/src/kernel/ipl09.nas}
  \caption{读取磁盘数据到内存}
  \label{sec:readfrag}
\end{listing}

\section{内存管理程序代码(memory.c)节选}

\begin{listing}[H]
  \inputminted[tabsize=2, firstline=68, lastline=78,
  linenos=true]{c}{../ZOS/src/kernel/memory.c}
  \caption{分配内存}
  \label{lst:alloc}
\end{listing}

\begin{listing}[H]
  \inputminted[tabsize=2, firstline=91, lastline=95,
  linenos=true]{c}{../ZOS/src/kernel/memory.c}
  \caption{确定采取何种方式释放内存}
  \label{lst:rw}
\end{listing}

\begin{listing}[H]
  \inputminted[tabsize=2, firstline=98, lastline=116,
  linenos=true]{c}{../ZOS/src/kernel/memory.c}
  \caption{前端可用}
  \label{lst:mem1}
\end{listing}

\begin{listing}[H]
  \inputminted[tabsize=2, firstline=118, lastline=127,
  linenos=true]{c}{../ZOS/src/kernel/memory.c}
  \caption{后端空闲}
  \label{lst:mem2}
\end{listing}

\begin{listing}[H]
  \inputminted[tabsize=2, firstline=128, lastline=141,
  linenos=true]{c}{../ZOS/src/kernel/memory.c}
  \caption{前端后端均被占用}
  \label{lst:mem3}
\end{listing}

\end{document} % 结束。不要动下面几行!

%%% Local Variables:
%%% mode: latex
%%% TeX-master: t
%%% End:
