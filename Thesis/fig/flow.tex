\documentclass[UTF8]{ctexart}
\usepackage{tikz,mathpazo}
\usetikzlibrary{shapes.geometric, arrows}
\begin{document}
\thispagestyle{empty}
 % 流程图定义基本形状
 \tikzstyle{startstop} = [rectangle, rounded corners, minimum width=3cm, minimum height=1cm,text centered, draw=black, fill=red!30]
 \tikzstyle{io} = [trapezium, trapezium left angle=70, trapezium right angle=110, minimum width=3cm, minimum height=1cm, text centered, draw=black, fill=blue!30]
 \tikzstyle{process} = [rectangle, minimum width=3cm, minimum height=1cm, text centered, draw=black, fill=orange!30]
 \tikzstyle{decision} = [diamond, minimum width=3cm, minimum height=1cm, text centered, draw=black, fill=green!30]
 \tikzstyle{arrow} = [thick,-&gt;,&gt;=stealth]
  
  \begin{tikzpicture}[node distance=2cm]
	   %定义流程图具体形状
	  \node (start) [startstop] {Start};
	  \node (in1) [io, below of=start] {Input};
	  \node (pro1) [process, below of=in1] {Process 1};
	  \node (dec1) [decision, below of=pro1, yshift=-0.5cm] {Decision 1};
	  \node (pro2a) [process, below of=dec1, yshift=-0.5cm] {Process 2a};
	  \node (pro2b) [process, right of=dec1, xshift=2cm] {Process 2b};
	  \node (out1) [io, below of=pro2a] {Output};
	  \node (stop) [startstop, below of=out1] {Stop};
	   
	    %连接具体形状
		\draw [arrow](start) -- (in1);
		\draw [arrow](in1) -- (pro1);
		\draw [arrow](pro1) -- (dec1);
		\draw [arrow](dec1) -- (pro2a);
		\draw [arrow](dec1) -- (pro2b);
		\draw [arrow](dec1) -- node[anchor=east] {yes} (pro2a);
		\draw [arrow](dec1) -- node[anchor=south] {no} (pro2b);
		\draw [arrow](pro2b) |- (pro1);
		\draw [arrow](pro2a) -- (out1);
		\draw [arrow](out1) -- (stop);
  \end{tikzpicture}

\end{document}
