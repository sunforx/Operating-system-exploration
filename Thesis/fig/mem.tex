\documentclass[11pt]{article}                                        
\usepackage[margin=1in]{geometry}                                    
\usepackage{longtable}
\usepackage[bitheight=6ex]{bytefield}
\usepackage{color}
\usepackage[bookmarksopen=trure]{hyperref}

\begin{document}
	\begin{bytefield}[bitheight=3\baselineskip]{40}
    \bitheader{0,7-8,15-16,23-24,31-32,40} \\


      % We have to do the \parbox explicitly in the next line because
      % \hyperlink typesets its argument in horizontal mode.
	\bitbox{8}{\hyperlink{protocol-version}{\parbox{\width}{\centering occupy}}} &
	\bitbox{16}{\hyperlink{packet-type-modifier}{free}} &
	\bitbox{8}{\hyperlink{packet-type-modifier}{release}} &
	\bitbox{8}{\hyperlink{subchannel}{occupy}} \\

      %\wordbox{1}{\hyperlink{source-connect}{source connection identifier}} \\

      %\wordbox{1}{\hyperlink{dest-connect}{destination connection identifier}} \\

      %\wordbox{1}{\hyperlink{msg-accept}{message acceptance criteria}} \\

      %\wordbox{1}{\hyperlink{heartbeat}{heartbeat}} \\

      % \bitbox{16}{\hyperlink{window}{window}} &
      % \bitbox{16}{\hyperlink{retention}{retention}}


  %\begin{rightwordgroup}{data \\ fields}
  %  \wordbox[lrt]{1}{%
  %    \parbox{0.6\width}{\centering (data content and format dependent on packet type and modifier)}} \\
  %  \skippedwords \\
  %  \wordbox[lrb]{1}{} 
  %\end{rightwordgroup}
  \end{bytefield}
\end{document}
