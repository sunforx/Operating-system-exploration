\documentclass{standalone}
    \usepackage{xeCJK}
    \usepackage{tikz}
    \usetikzlibrary{shapes.geometric, arrows.meta, arrows,positioning}
    
  \begin{document}
  \begin{tikzpicture}
    [auto,
    block/.style={rectangle, draw=black, thick,
    text width=10em,align=center, rounded corners,
    minimum height=4em},
    blank/.style={rectangle, thick,
    text width=10em,align=center, rounded corners,
    minimum height=4em},
    line/.style={draw, thick, shorten >=2pt, ->}]
  
    \matrix [column sep=5mm,row sep=7mm]
    {
        \node [block] (t1) {第一代操作系统\\真空管}; &
        \node [block] (t2) {第二代操作系统\\晶体管}; &
        \node [block] (t3) {第三代操作系统\\集成电路}; &
        \node [block] (t4) {第四代操作系统\\大规模集成电路}; &
        \node [block] (t5) {第五代操作系统\\移动计算机}; \\
        \node [blank] (s1) {人工操作}; &
        \node [blank] (s2) {读卡机和磁带机代替了一部分人工操作}; &
        \node [blank] (s3) {多道程序设计和分时操作系统技术出现}; &
        \node [blank] (s4) {计算机的体积缩小,计算机被普及,加入许多生活功能}; &
        \node [blank] (s5) {移动操作系统}; \\
    };
    \begin{scope}[every path/.style=line]
        \path (t1) -- (t2);
        \path (t2) -- (t3);
        \path (t3) -- (t4);
        \path (t4) -- (t5);
    \end{scope}
    \end{tikzpicture}
\end{document}
  
  %%% Local Variables:
  %%% mode: latex
  %%% TeX-master: t
  %%% End:
  