\documentclass{standalone}
    \usepackage{xeCJK}
    \usepackage{tikz}
    \usetikzlibrary{shapes.geometric, arrows.meta, arrows,positioning}
    
  \begin{document}
  \begin{tikzpicture}
    [auto,
    block/.style={rectangle, draw=black, thick,
    text width=10em,align=center, rounded corners,
    minimum height=4em},
    blank/.style={rectangle, thick,
    text width=10em,align=center, rounded corners,
    minimum height=4em},
    line/.style={draw, thick, shorten >=2pt, ->}]
  
    \matrix [column sep=5mm,row sep=7mm]
    {
        \node [block] (t0) {系统接口}; \\
        \node [block] (t2) [below=of t0]{命令行计算器}; 
        \node [block] (t1) [left=of t2]{文本阅读器}; 
        \node [block] (t3) [right=of t2]{音乐播放器}; 
        \node [block] (t4) [right=of t3]{图片阅读器}; \\
        \node [blank] (s1) [below=of t2, yshift=4cm]{命令行\\键盘输入\\现实字符串}; 
        \node [blank] (s2) [below=of t1, yshift=4cm]{命令行\\调用文件操作}; 
        \node [blank] (s3) [below=of t3, yshift=4cm]{命令行\\文件操作\\定时器}; 
        \node [blank] (s4) [below=of t4, yshift=4cm]{命令行\\文件操作}; \\
    };
    \begin{scope}[every path/.style=line]
        \path (t0) -- (t1);
        \path (t0) -- (t2);
        \path (t0) -- (t3);
        \path (t0) -- (t4);
    \end{scope}
    \end{tikzpicture}
\end{document}
  
  %%% Local Variables:
  %%% mode: latex
  %%% TeX-master: t
  %%% End:
  