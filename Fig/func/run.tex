\documentclass{standalone}
  \usepackage{xeCJK}
  \usepackage{tikz}
  \usetikzlibrary{shapes.geometric, arrows.meta, arrows,positioning}
  \newcommand*{\h}{\hspace{5pt}}% for indentation
  \newcommand*{\hh}{\h\h}% double indentation
  \tikzstyle{arrow} = [->,>=stealth]
\begin{document}
\begin{tikzpicture}
  [auto,
  block/.style={rectangle, draw=black, thick,
  text width=10em,align=center, rounded corners,
  minimum height=4em},
  block_left/.style ={rectangle, draw=black, thick, fill=white,
      text width=10em, text ragged, minimum height=4em, inner sep=6pt},
  line/.style={draw, thick, shorten >=2pt, ->}]

  \matrix [column sep=5mm,row sep=7mm]
  {
      \node [block](start){start}; 
      \node [block_left](init)[right=of start]{ 初始化:\\
        \hh init\_gdtidt() \\
        \hh init\_pic() \\
        \hh fifo32\_init() \\
        \hh init\_pit() \\
        \hh init\_keyboard() \\
        \hh enable\_mouse() \\
        \hh memman\_init() \\
        \hh task\_init() \\
      }; 
      \node [block](mem)[right=of init]{内存管理}; 
      \node [block](io) [above=of mem, yshift=1cm, xshift=0cm] {输入/输出}; 
      \node [block](mul)[below=of mem, yshift=-1cm, xshift=0cm] {多道程序设计}; \\
  };
  \begin{scope}[every path/.style=line]
    \path (start) -- (init);
    \path (init) -- (mem);
    \path (init) -- (io);
    \path (init) -- (mul);
  \end{scope}
\end{tikzpicture}

\end{document}

%%% Local Variables:
%%% mode: latex
%%% TeX-master: t
%%% End:
